%************************************************
\chapter{Experiment}\label{ch:experiment}
%************************************************
%XZ
%Collecting data
%System details

This chapter describes the main experiment that is conducted to evaluate both analyses.
We first explain the general research setup in \autoref{ch:Research-setup} that we use. 
Last, in \autoref{ch:system-detail} we present the hardware on which we execute the system.

\section{Research setup}\label{ch:Research-setup}

In the following we introduce the compression tool \textsc{XZ} in detail. Furthermore, we define the experiments and projects we use with 
the VaRA Tool Suite to analyze the system.

\subsection{Compression tool \textsc{XZ}}

The configurable system we analyze is the open source software system \textsc{XZ}, a command line tool written in \textsc{C} and a component of
\textsc{XZ Utils}\footnote{Visited at 02.04.2023 https://tukaani.org/xz/}.  
The primary compression algorithm that \textsc{XZ} currently uses is \textsc{LZMA2} to perform lossless data compression.
We use \textsc{XZ} \textsc{(XZ Utils)} version 5.5.0 with the commit hash "610dde15a88f12cc540424eb3eb3ed61f3876f74"

\subsection*{Configuration Space}
\textsc{XZ} offers us various features, due to the issue of combinatorial explosion \autoref{section:combinatorial-explosion} we 
have to select a subset of features so that we can still measure all possible configurations. We decided to select three features
that all should have a significant influence the performance of the system. We have two numeric features, \emph{compression level} and 
\emph{threads} and one binary feature \emph{extreme}. The value of \emph{compression level} ranges between \emph{0} and \emph{9} whereas \emph{threads} can either 
be \emph{0, 1, 2, 4} or \emph{8}. 

The feature \emph{compression level} sets the compression preset level, which influence the compression ratio.
This increases the necessary memory needed during compression and decompression, as well as the compression speed. 
This feature is encoded as an alternative group.

When \emph{extreme} is enabled \textsc{LZMA} uses a slower variant of the selected compression preset to achieve a potentially better compression ratio.
This feature is optional.

The feature \emph{threads} specifies the number of worker threads to use, whereas the option $0$ makes \textsc{XZ}  use as many threads as there are cores available.
Multithreading allows \emph{XZ} to split the input into different blocks and compress them independently of another.
This feature is optional, encoded as an alternative group.

%Feature Model

The number of configurations we can build using these three features is calculated as follows:

\begin{align}
    \lvert C \rvert &= \lvert \textit{compression level} \rvert \cdot \lvert \textit{extreme} \rvert \cdot \lvert \textit{threads} \rvert \\
    100 &= 10 \cdot 2 \cdot 5 \nonumber
\end{align}


\section{System detail}\label{ch:system-detail}