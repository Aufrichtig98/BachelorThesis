\section{Experiment operationalization}\label{ch:Research-setup}
This section describes the main experiment that is conducted to evaluate both analyses.

The configurable system we analyze is the open source software system \textsc{XZ}, a command line tool written in \textsc{C} and a component of
\textsc{XZ Utils}\footnote{Visited at 02.04.2023\\ \url{https://tukaani.org/xz/}}.  
The primary compression algorithm that \text{XZ} currently uses to perform lossless data compression is LZMA2.
We use \textsc{XZ} \textsc{(XZ Utils)} version 5.5.0\footnote{Commit hash "610dde15a88f12cc540424eb3eb3ed61f3876f74"}.

The file we compress is a geographic map encoded as a \textsc{JSON} file of size 203 MB\footnote{Visited at 03.04.2023\\ \url{https://github.com/simonepri/geo-maps/releases/latest/download/countries-land-1m.geo.json}},
which ensures that we spend a significant amount of time so that all features can be measured.

\subsection{Configuration Space}
%Why config space an features we are interested in
Due to the challenge of combinatorial explosion explained in \autoref{section:combinatorial-explosion}.
We have to select a subset of features \textsc{XZ} offers to still measure all possible configurations and ensure the accuracy of the black-box analysis. 

We selected three core features of \textsc{XZ} that should significantly influence the system's performance: 
two numeric features, \emph{compression level} and \emph{threads}, and one binary feature, \emph{extreme}. 
The value of \emph{compression level} ranges between \emph{0} and \emph{9} whereas \emph{threads} can either be \emph{0, 1, 2, 4} or \emph{8}. 
%Compression
The feature \emph{compression level} sets the compression preset level, which influences the compression ratio.
This increases the necessary memory needed during compression and decompression, as well as the compression speed. 
This feature is encoded as an alternative group.
%extreme
When \emph{extreme} is enabled \textsc{LZMA} uses a slower variant of the selected compression preset to achieve a potentially better compression ratio.
This feature is optional.
%threads
The feature \emph{threads} specifies the number of worker threads to use, whereas the option $0$ makes \textsc{XZ}  use as many threads as there are cores available.
Multithreading allows \emph{XZ} to split the input into different blocks and compress them independently of another.
This feature is optional, encoded as an alternative group.

%Feature Model

The number of configurations we can build using these three features is calculated as follows:

\begin{align}
    \lvert C \rvert &= \lvert \textit{compression level} \rvert \cdot \lvert \textit{extreme} \rvert \cdot \lvert \textit{threads} \rvert \\
    100 &= 10 \cdot 2 \cdot 5 \nonumber
\end{align}

To identify the \emph{Base} feature for \textsc{XZ}, we need to identify the time spent in compression no matter which features are selected. 
However, \textsc{XZ} does not allow us to select the mode compression without choosing a compression level. 
Therefore, as the \emph{Base} feature, we choose the configuration with the minimal time spent compressing, 
which is \emph{compression level 0} and \emph{threads 0}.