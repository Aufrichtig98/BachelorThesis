\section{Ground Truth}\label{sec:ground-truth}
In this section, we introduce our ground truth to establish that both black-box analysis 
and white-box analysis can analyze simple, configurable software systems. 

To do so, we design multiple configurable systems that test both analyses in different scenarios. 
Afterward, we evaluate as described in \autoref{ch:operationalization}. 
Since we design these systems ourselves, we have a baseline from which we manually build the {\perfInfluenceModel}s that we use for comparison. 
We design all these systems with a different focus in mind, such as a system that includes multicollinearity, 
to see to what extent they influence the analyses. 

Each system uses \emph{Simple Interaction} as the base system, however, they extend or change the system in different ways.

\paragraph{Simple Interaction}\label{ground-truth:Simple}
For our first system, we use the code from our previous example in \autoref{lst:performanceExample} and provide an additional feature model 
of the system in \autoref{fig:feature_abcd}. 
Since we use this as our base system, extending as needed to fit each
particular scenario, it is kept intentionally simple, only containing some interactions and no constraints.

\begin{figure}[h]
    \centering
    \includegraphics[scale=0.25]{gfx/Feature_ABCD.png}
    \caption{Feature model of \autoref{lst:performanceExample}.}
    \label{fig:feature_abcd}
\end{figure}


\paragraph{Else Clause}
The first modification is the addition of an \emph{else clause} to the if statement of feature \emph{A} in \autorefLine{line:feature_A_statement}. 
We encode this as follows:

\begin{minipage}{\linewidth}
    \begin{lstlisting}[language=C++,label={lst:else_case},escapechar=|]
if(A)
    fpcsc::sleep_for_secs(1); 
else
    fpcsc::sleep_for_secs(2);
    \end{lstlisting}
    \end{minipage}

We expect the white-box analysis to attribute the time spent in the \emph{else clause} to feature \emph{A}, if \emph{A} deselected. 
In contrast, 
the black-box analysis cannot differentiate between the \emph{else clause} that is executed when feature \emph{A} is deselected and the time spent in \emph{Base}.
We are interested in whether the white-box {\perfInfluenceModel} still identifies the influence of feature \emph{A} correctly, even though
the analysis attributed time to \emph{A} when it is deselected.

\paragraph{Function}\label{ground-truth:Function}
Another modification of our \emph{Simple Interaction} system is adding a function in which we spend time. 
In \autorefLine{line:feature_A} we call the function \emph{waste\_time(1)} instead of \emph{fpcsc::sleep\_for\_secs(1)}.

\begin{minipage}{\linewidth}
\begin{lstlisting}[language=C++,label={lst:function},escapechar=|]
void waste_time(int duration){
    fpcsc::sleep_for_secs(duration);
}
\end{lstlisting}
\end{minipage}

Both white-box and black-box analysis should be able to still correctly attribute the time spent in each feature.

\paragraph{Multicollinearity}\label{ground-truth:Multicollinearity}
Instead of modifying the system's code, we add a restriction to the feature model of \autoref{fig:feature_abcd}. 
We change feature \emph{B} from optional to mandatory, this changes the feature model and 
introduces multicollinearity into the system. 
We see the modified in \autoref{fig:multicollinearity}.
The white-box analysis should still be able to correctly identify the time spent in \emph{Base} and feature \emph{B}, 
while the black-box analysis should distribute the time spent between the features differently.

\begin{figure}[H]
    \centering
    \includegraphics[scale=0.25]{gfx/Multicollinearity.png}
    \caption{Feature model of the \emph{Multicollinearity} system.}
    \label{fig:multicollinearity}
\end{figure}

\paragraph{Shared Feature Variable}\label{ground-truth:Shared}
We now add an optional feature \emph{E} to the system. 
However, instead of encoding it like the previous features by specifying a separate variable, 
we modify the feature variable \emph{D} in \autorefLine{line:feature_declatation} to a \texttt{string de = "00"} 
for which the first character represents feature \emph{D} and the second character feature \emph{E}. If either is selected, 
we assign the character for that feature \emph{"1"}. We add another \emph{if statement} to spend time if \emph{E} is selected and encode this as follows:

\begin{minipage}{\linewidth}
\begin{lstlisting}[language=C++,label={lst:shared},escapechar=|]
std::string de = "00";

if(de[0] == '1')
    fpcsc::sleep_for_secs(2);

if(de[1] == '1')
    fpcsc::sleep_for_secs(1);
\end{lstlisting}
\end{minipage}

We expect the black-box analysis to still work as intended, but the white-box analysis to be inaccurate since both features \emph{D} and \emph{E} share the same feature variable. 
Each feature region of \emph{D} or \emph{E} should now be influenced by the feature interaction \emph{\{D, E\}}.