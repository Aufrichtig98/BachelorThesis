\section{Ground Truth}\label{sec:ground-truth}
In this section, we introduce a ground truth to establish that both black-box analysis from \autoref{ch:Blackbox} 
and white-box analysis from \autoref{ch:Whitebox} can analyze simple, configurable systems and identify weaknesses. 

To do so, we design multiple configurable systems that test both analyses in different scenarios. 
Afterward, we evaluate them using the research questions from \autoref{ch:operationalization}. 
Since we design these systems ourselves, we have a baseline that we use to build a \perfInfluenceModel for each system manually. 
We design all the systems with different focuses in mind, such as a system that includes multicollinearity, 
to see to what extent they influence the analyses. 
We investigate all the changes to \autoref{ground-truth:Simple} independent of another to pinpoint the reason if the analyses
are inaccurate, therefore, none of the changes are persistent.


\subsection{Simple interaction}\label{ground-truth:Simple}
For our first system, we use the code from our previous example in \autoref{lst:performanceExample} and provide an additional feature model 
of the system in \autoref{fig:feature_abcd}. 
We use this system as our base system, which we extend for each scenario; therefore it is the simplest one, 
only containing some interactions and no constraints.

\begin{figure}[h]
    \centering
    \includegraphics[scale=0.55]{gfx/Feature_ABCD.png}
    \caption{Feature model of \autoref{lst:performanceExample}.}
    \label{fig:feature_abcd}
\end{figure}


\paragraph{Else clause}
The first modification we do is adding an \emph{else clause} to the if statement of feature \emph{A} in \autorefLine{line:feature_A_statement}

\begin{minipage}{\linewidth}
    \begin{lstlisting}[language=C++,label={lst:else_case},escapechar=|]
if(A)
    fpcsc::sleep_for_secs(1); 
else
    fpcsc::sleep_for_secs(2);
    \end{lstlisting}
    \end{minipage}

We expect the white-box analysis to attribute the time spent in the else case to feature \emph{A} 
since it is a part of the feature region of feature \emph{A}. 
In contrast, 
the black-box analysis cannot differentiate between the else clause that is executed when feature \emph{A} is unselected and the time spent in \emph{Base}.

\paragraph{Function}\label{ground-truth:Function}
The next modification of our basic system is adding a function to the system in which we spent time. 
In \autorefLine{line:feature_A} we call the function \emph{waste\_time(1)} instead of \emph{fpcsc::sleep\_for\_secs(1)}, in which we spent time a second.

\begin{minipage}{\linewidth}
\begin{lstlisting}[language=C++,label={lst:function},escapechar=|]
void waste_time(int duration){
    fpcsc::sleep_for_secs(duration);
    }
\end{lstlisting}
\end{minipage}

This modification should not affect the black-box analysis since the overall time spent in each feature does not change. 
We would like to see if the white-box analysis can still attribute the time correctly to the feature.

\paragraph{Multicollinearity}\label{ground-truth:Multicollinearity}
Instead of modifying the system's code, we add a restriction to the feature model of \autoref{fig:feature_abcd}. 
We change feature \emph{B} from optional to mandatory. This introduces multicollinearity into the system. 
The white-box analysis should still be able to identify correctly the time spent in \emph{Base} and feature \emph{B}, 
while the black-box analysis should distribute the time spent between \emph{Base} and \emph{B} differently.

\paragraph{Shared Feature Variable}\label{ground-truth:Shared}
We now add an optional feature \emph{E} to the system. 
However, instead of encoding it like the previous features by specifying a separate variable, 
we modify the feature variable \emph{D} in \autorefLine{line:feature_declatation} to a \emph{string de = "00"} 
for which the first character represents feature \emph{D} and the second character \emph{E}. If either is selected, 
we assign the character for that feature \emph{"1"}. We add another if statement to spend time when \emph{E} is selected and encode this as follows:

\begin{minipage}{\linewidth}
\begin{lstlisting}[language=C++,label={lst:shared},escapechar=|]
std::string de = "00";

if(de[0] == '1')
    fpcsc::sleep_for_secs(2);

if(de[1] == '1')
    fpcsc::sleep_for_secs(1);
\end{lstlisting}
\end{minipage}

We expect the black-box analysis to work still as intended. 
However, the white-box analysis to be inaccurate since both features \emph{D} and \emph{E} share the same feature variable. 
Each feature region of \emph{D} or \emph{E} should now be influenced by the feature interaction \emph{\{D, E\}}.