%************************************************
\chapter{Evaluation}\label{ch:evaluation}
%************************************************
\lstset{style=myStyle}

This chapter evaluates the thesis's core claims.  
We present the results for both the ground truth systems and \textsc{XZ} in \autoref{sec:results}.
Afterward, we discuss the results in \autoref{sec:discussion}. At the end in \autoref{sec:threats}, we explain the threads of validity for the results.

\section{Results}\label{sec:results}

%To Review
In this section, we present the results of the ground truth and of \textsc{XZ}. 
All values are in seconds and are rounded to 3 decimal places, this should not influence the comparability of our models, since in comparison
to the overall time spent inside the system this influence is insignificant. Due to that, for the white-box we also discard all features
and feature interaction for which the summed up time spent in these regions is less than one millisecond.

\subsection{Ground Truth Results}
We now proceed with evaluating the ground truth systems \emph{Simple Interactions}, \emph{Else Clause}, \emph{Function}, \emph{Multicollinearity} 
and \emph{Shared Feature Variable}, which we introduced in \autoref{sec:ground-truth} and present the resulting {\perfInfluenceModel}s.

\subsubsection*{Simple Interaction}

\begin{table}[H]
    \centering
    \begin{tabular}{lrrrrrrr}
    \toprule
    $\Pi$    & \emph{Base} & \emph{A} & \emph{B} & \emph{C} & \emph{D} & \emph{A} $\land$ \emph{B} & \emph{C} $\land$ \emph{D}  \\ \midrule
    Baseline & 2    & 1 & 2 & 1 & 2 & 2           & 0            \\
    Black-box & 2    & 1 & 2 & 1 & 2 & 2           & 0           \\
    White-box & 2    & 1 & 2 & 1 & 2 & 2           & 0           \\ \bottomrule
    \end{tabular}  
    \caption{Direct comparison between the baseline, black-box and white-box {\perfInfluenceModel}s for \emph{Simple Interaction}}\label{aggr:results-simple-interaction}
\end{table}

The \emph{Simple Interaction} system is designed to test if both analyses can identify interactions between features, 
as we can see in \autoref{aggr:results-simple-interaction} both {\perfInfluenceModel}s were able to identify these interactions.

\subsubsection*{Else Clause}

\begin{table}[H]
    \centering
    \begin{tabular}{lrrrrrrr}
    \toprule
    $\Pi$    & \emph{Base} & \emph{A} & \emph{B} & \emph{C} & \emph{D} & \emph{A} $\land$ \emph{B} & \emph{C} $\land$ \emph{D}  \\ \midrule
    Baseline & 4    & -1 & 2 & 1 & 2 & 2           & 0            \\
    Black-box & 4    & -1 & 2 & 1 & 2 & 2           & 0           \\
    White-box & 4    & -1 & 2 & 1 & 2 & 2           & 0           \\ \bottomrule
    \end{tabular}  
    \caption{Direct comparison between the baseline, black-box and white-box {\perfInfluenceModel}s for \emph{Else Clause}}\label{aggr:results-else-clause}
\end{table}

The \emph{Else Clause} system is designed to see how both analysis attribute the time spent in the else clause, as expected, we see in
\autoref{aggr:results-else-clause} that they were both able to attribute the time correctly.

\subsubsection*{Function}

\begin{table}[H]
    \centering
    \begin{tabular}{lrrrrrrr}
    \toprule
    $\Pi$    & \emph{Base} & \emph{A} & \emph{B} & \emph{C} & \emph{D} & \emph{A} $\land$ \emph{B} & \emph{C} $\land$ \emph{D}  \\ \midrule
    Baseline & 2    & 1 & 2 & 1 & 2 & 2           & 0            \\
    Black-box & 2   & 1 & 2 & 1 & 2 & 2           & 0            \\
    White-box & 2   & 1 & 2 & 1 & 2 & 2           & 0            \\ \bottomrule
    \end{tabular}  
    \caption{Direct comparison between the baseline, black-box and white-box {\perfInfluenceModel}s for \emph{Function}}\label{aggr:results-function}
\end{table}

In the \emph{Function} system, we test how the analyses handle it when we spend time in a different function than the main function. 
We see in \autoref{aggr:results-function} that both analyses were still able to attribute the time correctly.

\subsubsection*{Multicollinearity}

\begin{table}[H]
    \centering
    \begin{tabular}{lrrrrrrr}
    \toprule
    $\Pi$    & \emph{Base} & \emph{A} & \emph{B} & \emph{C} & \emph{D} & \emph{A} $\land$ \emph{B} & \emph{C} $\land$ \emph{D}  \\ \midrule
    Baseline & 2    & 1 & 2 & 1 & 2 & 2           & 0            \\
    Black-box & 4   & 3 & 0 & 1 & 2 & 0           & 0            \\
    White-box & 2   & 1 & 2 & 1 & 2 & 2           & 0            \\ \bottomrule
    \end{tabular}  
    \caption{Direct comparison between the baseline, black-box and white-box {\perfInfluenceModel}s for \emph{Multicollinearity}}
    \label{aggr:results-mullticollinearity}
\end{table}

For the \emph{Multicollinearity} system, we change feature \emph{B} from an optional to a mandatory feature, 
for which we expect the black-box analysis to be unable to attribute the time for the affected features accurately. 
As we can see in \autoref{aggr:results-mullticollinearity}, for the features \emph{Base}, \emph{A}, \emph{B} and 
the interaction \emph{A} $\land$ \emph{B} the black-box is inaccurate, whereas the white-box remains unaffected.

\subsubsection*{Shared Feature Variables}

\begin{table}[H]
    \centering
    \begin{tabular}{lrrrrrrrr}
    \toprule
    $\Pi$    & \emph{Base} & \emph{A} & \emph{B} & \emph{C} & \emph{D} & \emph{E}& \emph{A} $\land$ \emph{B} & \emph{C} $\land$ \emph{D}  \\ \midrule
    Baseline & 2    & 1 & 2 & 1 & 2 & 1 & 2           & 0            \\
    Black-box & 4   & 3 & 0 & 1 & 2 & 1 & 0           & 0            \\
    White-box  &  6.001 &  0.0 &  0.0 &  0.0 &  0.0 &  0.0 &     0.0 &     0.0\\\bottomrule
    \end{tabular}  
    \caption{Direct comparison between the baseline, black-box and white-box {\perfInfluenceModel}s for \emph{Shared Feature Variables}}
    \label{aggr:results-shared-feature}
\end{table}

For the \emph{Shared Feature Variables}, we encoded two features \emph{D}, and \emph{E} as a single string. We expected that the white-box analysis is unable
to be unable to distinguish between these features and instead encode them as one. 
However, unexpectedly we did not identify any features besides the base feature during the analysis.

\subsection{Experiment Results}

We now present the {\perfInfluenceModel}s for \textsc{XZ}. The {\perfInfluenceModel} build from the black-box analysis can be seen
in the Appendix, where we split the table in two tables, where in \autoref{table:BB-XZ-noExtreme} the feature \emph{Extreme} in 
deselected and in \autoref{table:BB-XZ-Extreme} \emph{Extreme} is selected. The {\perfInfluenceModel} build from the white-box analysis 
can be seen in %Whitebox.

\subsection{Results Research Questions}
We now present our results for the research questions, for both we discard feature interactions that have an influence of 0 
across all {\perfInfluenceModel}s for the same system.

\subsection*{Results RQ1}

We now present the results for research question 1 by evaluating the ground truth systems using the {\perfInfluenceModel} previously
shown.

\subsubsection*{Simple Interaction}

\begin{table}[H]
\begin{minipage}{.5\linewidth}
    \centering
    \begin{tabular}{lrrrrrr}    \toprule
    $error$     & \emph{Base} & \emph{A} & \emph{B} & \emph{C} & \emph{D} & \emph{A} $\land$ \emph{B}   \\ \midrule
    Black-box & 0 & 0 & 0 & 0 & 0 & 0      \\
    White-box & 0 & 0 & 0 & 0 & 0 & 0      \\ \bottomrule
    \end{tabular}
    \caption{Respective \emph{error} scores for white-box and black-box {\perfInfluenceModel}s for the \emph{Simple Interaction} system.}
    \label{rq1:simple-interaction}
\end{minipage}%
\hspace{7mm}
\begin{minipage}{.37\linewidth}
    \centering
    \begin{tabular}{lr}
        \toprule
                  & $\overline{error}$   \\ \midrule
        Black-box & 0              \\
        White-box & 0              \\ \bottomrule
        \end{tabular}  
        \caption{$\overline{error}$ score for white-box and black-box for the \emph{Simple Interaction} system}
        \label{rq1:simple-interaction-mean}
    \end{minipage}
\end{table}

In \autoref{rq1:simple-interaction} we see the \emph{error} score for each 
feature and feature interaction which is $0$ for both analyses for the \emph{Simple Interaction} system. 
Furthermore, \autoref{rq1:simple-interaction-mean} shows the $\overline{error}$ score which is $0$ for both analyses.

\subsubsection*{Else Clause}

\begin{table}[H]
    \begin{minipage}{.5\linewidth}
        \centering
        \begin{tabular}{lrrrrrr}    \toprule
        $error$    & \emph{Base} & \emph{A} & \emph{B} & \emph{C} & \emph{D} & \emph{A} $\land$ \emph{B}   \\ \midrule
        Black-box & 0 & 0 & 0 & 0 & 0 & 0      \\
        White-box & 0 & 0 & 0 & 0 & 0 & 0      \\ \bottomrule
        \end{tabular}
        \caption{Respective \emph{error} scores for white-box and black-box {\perfInfluenceModel}s for the \emph{Else Clause} system.}
        \label{rq1:else-clause}
    \end{minipage}%
    \hspace{7mm}
    \begin{minipage}{.37\linewidth}
        \centering
        \begin{tabular}{lr}
            \toprule
                      & $\overline{error}$   \\ \midrule
            Black-box & 0              \\
            White-box & 0              \\ \bottomrule
            \end{tabular} 
            \caption{$\overline{error}$ score for white-box and black-box for the \emph{Else Clause} system}
            \label{rq1:else-clause-mean}
        \end{minipage}
    \end{table}
    
In \autoref{rq1:else-clause} we see the \emph{error} for each feature and feature interaction for both white-box and black-box 
analyses for the \emph{Else Clause} system. The $\overline{error}$ can be seen in \autoref{rq1:else-clause-mean} and 
both \emph{error} and $\overline{error}$ scores are $0$.

\subsubsection*{Function}

\begin{table}[H]
    \begin{minipage}{.5\linewidth}
        \centering
        \begin{tabular}{lrrrrrr}    \toprule
        $error$     & \emph{Base} & \emph{A} & \emph{B} & \emph{C} & \emph{D} & \emph{A} $\land$ \emph{B}   \\ \midrule
        Black-box & 0 & 0 & 0 & 0 & 0 & 0      \\
        White-box & 0 & 0 & 0 & 0 & 0 & 0      \\ \bottomrule
        \end{tabular}
        \caption{Respective \emph{error} scores for white-box and black-box {\perfInfluenceModel}s for the \emph{Function} system.}
        \label{rq1:function}
    \end{minipage}%
    \hspace{7mm}
    \begin{minipage}{.37\linewidth}
        \centering
        \begin{tabular}{lr}
            \toprule
                      & $\overline{error}$   \\ \midrule
            Black-box & 0              \\
            White-box & 0              \\ \bottomrule
            \end{tabular}
            \caption{$\overline{error}$ score for white-box and black-box for the \emph{Function} system}
            \label{rq1:function-mean}
        \end{minipage}
    \end{table}

In \autoref{rq1:function} we present the $error$ scores for the \emph{Function} system, for which all are $0$. In addition, 
\autoref{rq1:function-mean} shows a $\overline{error}$ score of $0$ for the white-box and black-box analyses.
    
    \subsubsection*{Multicollinearity}

    \begin{table}[H]
        \begin{minipage}{.5\linewidth}
            \centering
            \begin{tabular}{lrrrrrr}    \toprule
            $error$    & \emph{Base} & \emph{A} & \emph{B} & \emph{C} & \emph{D} & \emph{A} $\land$ \emph{B}   \\ \midrule
            Black-box & 1 & 2 & 1 & 0 & 0 & 1      \\
            White-box & 0 & 0 & 0 & 0 & 0 & 0      \\ \bottomrule
            \end{tabular}
            \caption{Respective \emph{error} scores for white-box and black-box {\perfInfluenceModel}s for the \emph{Multicollinearity} system.}
            \label{rq1:multicollinearity}
        \end{minipage}%
        \hspace{7mm}
    \begin{minipage}{.37\linewidth}
        \centering
        \begin{tabular}{lr}
            \toprule
                        & $\overline{error}$   \\ \midrule
            Black-box & $\frac{5}{6}$              \\
            White-box & 0              \\ \bottomrule
            \end{tabular}
            \caption{$\overline{error}$ score for white-box and black-box for the \emph{Multicollinearity} system}
            \label{rq1:multicollinearity-mean}
        \end{minipage}
    \end{table}

In \autoref{rq1:multicollinearity}, we see the \emph{error} scores for both white-box and black-box analyses. 
The score for the white-box analysis remains the same as before, whereas for the black-box we can see a different \emph{score} for the features \emph{Base}, 
\emph{A}, and the feature interaction \emph{A} $\land$ \emph{B}. 
Therefore, the $\overline{error}$ score presented in \autoref{rq1:multicollinearity-mean} for the black-box analyses is $\frac{5}{6}$. 
However, the score for the white-box analysis remains $0$. 

    \subsubsection*{Shared Feature Variable} %TO DO WHITE BOX 

        \begin{table}[H]
            \begin{minipage}{.5\linewidth}
                \centering
                \begin{tabular}{lrrrrrr}    \toprule
                $error$    \emph{Base} & \emph{A} & \emph{B} & \emph{C} & \emph{D} & \emph{E} & \emph{A} $\land$ \emph{B}   \\ \midrule
                Black-box & 1 & 2 & 1 & 0 & 0 & 0       \\
                White-box & 0 & 0 & 0 & 0 & 0 & 0      \\ \bottomrule
                \end{tabular}
                \caption{Respective \emph{error} scores for white-box and black-box {\perfInfluenceModel}s for the \emph{Shared Feature Variable} system.}
                \label{rq1:shared-feature-variable}
            \end{minipage}%
            \hspace{7mm}
        \begin{minipage}{.37\linewidth}
            \centering
            \begin{tabular}{lr}
                \toprule
                            & $\overline{error}$   \\ \midrule
                Black-box & $\frac{5}{6}$              \\
                White-box & 0              \\ \bottomrule
                \end{tabular}
                \caption{$\overline{error}$ score for white-box and black-box for the \emph{Shared Feature Variable} system}
                \label{rq1:shared-feature-variable-mean}
            \end{minipage}
        \end{table}

%%%%%%%%%%%%%%%%%%%%%%%%%%%%%%%%%%%%%%%%%%%%%
% RQ 2 
%%%%%%%%%%%%%%%%%%%%%%%%%%%%%%%%%%%%%%%%%%%%%

\subsection{Results RQ2}

We now present the results for research question 1 by evaluating the ground truth systems and the experiment using the {\perfInfluenceModel} 
previously shown.

\subsubsection*{Simple Interaction}

\begin{table}[H]
    \centering
    \begin{tabular}{lrrrrrr}    \toprule
               & \emph{Base} & \emph{A} & \emph{B} & \emph{C} & \emph{D} & \emph{A} $\land$ \emph{B}   \\ \midrule
    $similarity$ & 0 & 0 & 0 & 0 & 0 & 0      \\ \bottomrule
    \end{tabular}
    \caption{Respective $similarity$ score for each feature and a $\overline{similarity}$ score of $0$ for the \emph{Simple Interaction} system.}
    \label{rq2:simple-interaction}
\end{table}

In \autoref{rq2:simple-interaction}, we see the \emph{similarity} scores for the \emph{Simple Interaction} system with a 
$\overline{similarity}$ score of $0$

\subsubsection*{Else Clause}

\begin{table}[H]
    \centering
    \begin{tabular}{lrrrrrr}    \toprule
               & \emph{Base} & \emph{A} & \emph{B} & \emph{C} & \emph{D} & \emph{A} $\land$ \emph{B}   \\ \midrule
    $similarity$ & 0 & 0 & 0 & 0 & 0 & 0      \\ \bottomrule
    \end{tabular}
    \caption{Respective $similarity$ score for each feature and a $\overline{similarity}$ score of $0$ for the \emph{Else Clause} system.}
    \label{rq2:else-clause}
\end{table}

The $similarity$ score for the \emph{Else Clause} system can be seen in \autoref{rq2:else-clause}, for which the 
$\overline{error}$ is $0$.

\subsubsection*{Function}

\begin{table}[H]
    \centering
    \begin{tabular}{lrrrrrr}    \toprule
               & \emph{Base} & \emph{A} & \emph{B} & \emph{C} & \emph{D} & \emph{A} $\land$ \emph{B}   \\ \midrule
    $similarity$ & 0 & 0 & 0 & 0 & 0 & 0      \\ \bottomrule
    \end{tabular}
    \caption{Respective $similarity$ score for each feature and a $\overline{similarity}$ score of $0$ for the \emph{Function} system.}
    \label{rq2:function}
\end{table}
    
The \autoref{rq2:function} shows the $similarity$ scores for the \emph{Function} system with a $\overline{similarity}$ score of $0$.

\subsubsection*{Multicollinearity}

\begin{table}[H]
    \centering
    \begin{tabular}{lrrrrrr}    \toprule
               & \emph{Base} & \emph{A} & \emph{B} & \emph{C} & \emph{D} & \emph{A} $\land$ \emph{B}   \\ \midrule
    $similarity$ & $\frac{2}{12}$ & $ \frac{2}{12}$ &  $\frac{2}{12}$ & 0 & 0 &  $\frac{2}{12}$      \\ \bottomrule
    \end{tabular}
    \caption{Respective $similarity$ score for each feature and a $\overline{similarity}$ score of $\frac{1}{9}$ for the \emph{Multicollinearity} system.}
    \label{rq2:multicollinearity}
\end{table}

The $similarity$ scores for the \emph{Multicollinearity} system are in \autoref{rq2:multicollinearity}, we see that the $similarity$
scores for the features \emph{Base}, \emph{A}, \emph{B} and feature interaction \emph{A} $\land$ \emph{B} changed. Here the 
value $12$ is the time of the black-box {\perfInfluenceModel} $\Pi_{BB}$. The $\overline{similarity}$ score is $\frac{1}{9}$.


\subsubsection*{Shared Feature Variable} %TO DO WHITE BOX 

\begin{table}[H]
    \centering
    \begin{tabular}{lrrrrrrr}    \toprule
               & \emph{Base} & \emph{A} & \emph{B} & \emph{C} & \emph{D} & \emph{E} & \emph{A} $\land$ \emph{B}   \\ \midrule
    $similarity$ & 0 & 0 & 0 & 0 & 0 & 0      \\ \bottomrule
    \end{tabular}
    \caption{Respective $similarity$ score for each feature and a $\overline{similarity}$ score of $\frac{1}{9}$ for the \emph{Shared Feature Variable} system.}
    \label{rq2:shared-feature-variable}
\end{table}

\subsubsection*{XZ Results}

\section{Discussion}\label{sec:discussion}

%Intro into the section
In this section, we discuss the results of both analyses. We will first discuss the resulting {\perfInfluenceModel}s 
for both the ground truth systems and \textsc{XZ} and afterwards the results of the research questions.

%Analysis correct for the first three systems
We first take a look at the {\perfInfluenceModel} created by both analyses, 
for which we start by inspecting the {\perfInfluenceModel}s for the ground truth system \emph{Simple Interactions}, 
\emph{Else Cause}, and \emph{Function}, as expected the {\perfInfluenceModel}s for all three systems are identical to another. 
This means that both analyses accurately identified the influence of all features and feature interactions for simple systems.
These results are in line with the results for the research questions for these systems. For both analyses
we have a $\overline{error}$ score of $0$ which confirms that our {\perfInfluenceModel} are identical to the baseline and therefore 
accurate. Hence, if both analyses {\perfInfluenceModel} are identical to the baseline, they are also identical themselves,
which the $\overline{similarity}$ score of $0$ confirms.


%Multicollinearity
For the \emph{Multicollinearity} system, our hypothesis holds the white-box analysis still identifies the time spent inside the multicollinear features correctly, whereas the {\perfInfluenceModel} of the black-box analysis is severely different. The first change that we see is that due to feature \emph{B} being mandatory, the feature \emph{A} and the feature interaction $A \land B$ are perfectly Multicollinear since feature \emph{B} 
is always selected together with feature \emph{A}. 
This is detected when we decide which terms to add to our multiple linear regression model by \refAlgorithm{alg:vif_iterative} 
and thus \emph{A} $\land$ \emph{B} is not added to the model. In addition, due to multicollinearity, the time distribution between features in the 
{\perfInfluenceModel} changed. While features \emph{C} and \emph{D} are still correctly attributed, 
we have a change for features \emph{Base} with an increase of 2 seconds, \emph{A} with an increase of 2 seconds, 
and \emph{B} with a decrease of 2 seconds.
%%%
We assume that the time spent in \emph{B} is attributed to \emph{Base} since for the black-box analysis, both features are mandatory and
thus, in every configuration we measure, which makes them indistinguishable. 
The 2 seconds spent in the interaction \emph{A} $\land$ \emph{B} is attributed to \emph{A} due to the perfect multicollinearity 
between these features. 
%%%
Now when inspecting the \emph{error} scores, this difference shows for the features affected by multicollinearity, 
which results in an average percentage error of $\frac{5}{6}$, 
which indicates a high average error when using black-box model for multicollinear systems.
For this system we have a $\overline{similarity}$ score of $frac{1}{9}$, which means that each feature got an average 
difference of $11\%$ compared to the overall time of $12$ seconds which significant difference for these models and 
means that they are not interchangeable for systems that contain multicollinearity.

Next, we inspect the results for the \emph{Shared Feature Variables} system. 
As expected, the black-box analysis identified the influence of all features correctly. 
However, for the white-box analysis, something unexpected happened. 
We expected that the white-box analysis could not correctly identify the influence of feature \emph{D} and \emph{E} since they share one 
feature variable. However, we were unable to identify and feature. When we inspect \autoref{table:WB-Shared-Feature}, 
we see that for each configuration we analyzed \textsc{VaRA} only found the \emph{Base} feature, 
although we did not change the feature variables for \emph{A}, \emph{B}, \emph{C}.

\section{Threats to Validity}\label{sec:threats}

In this section, we discuss threats to internal and external validity of our evaluation.

A threat to the internal validity is that during the execution of the system, 
external influences can affect the performance of the server node and, therefore our measurements. 
To minimize this, we ensured that the only thing being executed on the server node during the measurements is our system. 
In addition, we repeated every measurement $30$ times to reduce the impact of outliers.  

Another threat to the internal validity is the evaluation of the black-box data. 
For the evaluation we used well known library such as \emph{sklearn} and \emph{statsmodels} 
to minimize the chance for an implementation error by our side.

A threat to the external validity is that the systems we choose favor one analysis over the other. 
To prevent this we build different systems in our ground truth that tested both the black-box and white-box analysis 
in various manners.