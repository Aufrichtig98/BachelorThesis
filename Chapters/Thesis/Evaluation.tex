%************************************************
\chapter{Evaluation}\label{ch:evaluation}
%************************************************
\lstset{style=myStyle}

This chapter evaluates the thesis core claims.  
We begin with presenting the results for both the ground truth systems and \textsc{XZ} in \autoref{sec:results}.
Afterward in \autoref{sec:discussion} we discuss the results. At the end in \autoref{sec:threats}, we explain the threads of validity for the results.

\section{Results}\label{sec:results}

In this section, we present the results of the ground truth and of \textsc{XZ}. All values are in seconds which been rounded to 3 decimal places.

\subsection*{Ground Truth Results}
We now proceed with evaluating the different systems, \textsc{Simple Interactions}, \textsc{Else Clause}, \textsc{Function}, \textsc{Multicollinearity} 
and \textsc{Shared Feature Variable}, that we introduce in \autoref{sec:ground-truth}

\subsubsection*{Simple Interaction Results}
%Baseline and Blackbox
We start by constructing the baseline 
{\perfInfluenceModel} for this system as shown in \autoref{equ:performanceExamplePIMBaseline} and build the {\perfInfluenceModel} for the black-box analysis:

\begin{table}[H]
    \centering
    \input{Tables/BlackBox/GTBasic.txt}
    \caption{Black-box {\perfInfluenceModel} for simple interaction}
\end{table}

For the white-box analysis with \autoref{lst:performanceExample} and \autoref{fig:feature_abcd} we build \autoref{alg:xml-abcd} 
in which we declare all the feature variables of \textsc{Simple Interactions}. 
We analyze \autoref{lst:performanceExample} using our white-box analysis, this produces a TEF report file in \autoref{rep:tef-abcd}.
Afterward we use this report to calculate the time spend in each feature region described as in \autoref{math:time} and \autoref{math:coefficients}.
This produces the following file in \autoref{alg:aggr-tef-abcd}, from which we build the following {\perfInfluenceModel}:

\begin{table}[H]
    \centering
    \begin{tabular}{lrrrrrrr}
    \toprule
    $\Pi_{WB}$    & Base & A & B & C & D & A $\land$ B & C $\land$ D  \\
    \midrule
    Simple Interaction &   2.0 &  1.0 &  2.0 &  1.0 &  2.0 &   2.0 &  0.0 \\
    \bottomrule
    \end{tabular}
    \caption{White-box {\perfInfluenceModel} for simple interaction}
\end{table}

We group our results in \autoref{aggr:results-simple-interaction} to compare them with each other.

\begin{table}[H]
    \centering
    \begin{tabular}{lrrrrrrr}
    \toprule
    $\Pi$    & Base & A & B & C & D & A $\land$ B & C $\land$ D  \\ \midrule
    Baseline & 2    & 1 & 2 & 1 & 2 & 2           & 0            \\
    Black-box & 2    & 1 & 2 & 1 & 2 & 2           & 0           \\
    White-box & 2    & 1 & 2 & 1 & 2 & 2           & 0           \\ \bottomrule
    \end{tabular}  
    \caption{Direct comparison between the baseline, black-box and white-box {\perfInfluenceModel} for \autoref{lst:performanceExample}}\label{aggr:results-simple-interaction}
\end{table}

Out of \autoref{aggr:results-simple-interaction} we now calculate $\overline{error}$ and %TO DO SIMILARITY
for both white-box and black-box:

\begin{table}[H]
\begin{minipage}{.5\linewidth}
    \centering
    \begin{tabular}{lr}
    \toprule
    RQ1     & $\overline{error}$         \\ \midrule
    Black-box & 0              \\
    White-box & 0              \\ \bottomrule
    \end{tabular}  
    \caption{Results RQ1}
\end{minipage}%
\begin{minipage}{.5\linewidth}
    \centering
    \begin{tabular}{lr}
        \toprule
        RQ2     & $similarity$    \\ \midrule
        Black-box$_\text{WB}$ & 0              \\
        White-box$_\text{BB}$ & 0              \\ \bottomrule
        \end{tabular}  
        \caption{Results RQ2}
\end{minipage} 
\end{table}

\subsubsection*{Else Clause Results}

The results for the \textsc{Else Clause} system are as follows. The \perfInfluenceModel of the baseline is as follows:

\begin{table}[H]
    \centering
    \begin{tabular}{lrrrrrrr}
    \toprule
    $\Pi_{Baseline}$    & Base & A & B & C & D & A $\land$ B & C $\land$ D  \\
    \midrule
    Baseline &   4 &  -1 &  2 &  1 &  2 &   2 &  0 \\
    \bottomrule
    \end{tabular}
    \caption{Baseline {\perfInfluenceModel} for \emph{Else Clause}}
\end{table}

We attributed the time spend in the else case to the base feature, since it represents the time spent in the system when no feature is selected.
Therefore, when selecting feature \emph{A} the system gets faster by one second. 

\begin{table}[H]
    \centering
    \input{Tables/BlackBox/GTElse.txt}
\end{table}


To build the {\perfInfluenceModel} for the white-box we have to take a closer look at 


\section{Discussion}\label{sec:discussion}

In this section, discuss your results.

\section{Threats to Validity}\label{sec:threats}

In this section, discuss the threats to internal and external validity.
