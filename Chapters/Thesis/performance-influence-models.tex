\chapter{Performance-influence models}\label{ch:performance-influence-models}

Both of white-box and black-box approach produce data that differ from each other, so we cannot compare them directly, for 
this reason, we intoduce performance-influence models. We generated such a model for each dataset generatey by both white-box and black-box, then
we are able to compare and analyze both model. We use the formal definitons from the paper "Performance-Influence Models for
Highly Configurable Systems" by Sven Apel et al. \cite{Performance-influence-models-for-highly-configurable-systems}.

A feature performance model is a polynomial that consists of several terms, each term representing either a feature or an interaction
of features and the influence they have onto the system itself. The sum of all terms represent the time the performance model predicts
given a configuration of features. All our features are encoded as binary features, meaning they can either be turned on or off. Formally, 
let $\mathcal{O}$ be the set of all configuration options and $\mathcal{C}$ the set of all configuration, then  let $\emph{c} \in \mathcal{C}$ be
a function $\emph{c} \textit{ : } \mathcal{O} \implies \mathbb{R}$ that assigns either $0$ or $1$ to each binary option. If we select a feature
$o$, then $\emph{c}(\emph{o}) = 1$ holds, and if we deselect a feature $\emph{c}(\emph{o}) = 0$. In general, a performance-influence
model is a function $\Pi$ that maps configurations $mathcal$ to an estimated time preditcion, therefore $\Pi \textit{ : } \mathcal{C} \implies \mathbb{R}$.

We distinguish between single features $o$ denoted as $\phi_o$ and feature interactions $i ... j$ denoted as $\Phi_{i...j}$. With these
definitions, we can now define how a performance-influence model formally looks like:

\begin{gather}
    \Pi = \beta_0 + \sum_{i \in \mathcal{O}} \phi_i(c(i)) + \sum_{i...j \in \mathcal{O}} \Phi_{i...j}(c(i)...c(j))
\end{gather}

While $\beta_0$ denotes the base performance, which refers to the time taken by the system regardless of configuration, $\sum_{i \in \mathcal{O}} \phi_i(c(i))$
is the sum of each individuall feature and $\sum_{i...j \in \mathcal{O}} \Phi_{i...j}(c(i)...c(j))$ is the sum of each feature interaction.


\makeatletter
\def\BState{\State\hskip-\ALG@thistlm}
\makeatother

\begin{algorithm}
\caption{Feature Interaction \label{alg:performanceExample}}
\begin{algorithmic}[1]
\Procedure{Feature Interaction}{}
\State $\textit{a,b,c,d} \gets \textit{True or False}$
\State $\textit{sleep\_for\_seconds(1)}$ $\textit{\#Lets the system sleep for 1 second}$
\If{$a$} 
    \State $\textit{sleep\_for\_seconds(1)}$
\EndIf
\If{$b$} 
    \State $\textit{sleep\_for\_seconds(2)}$
\EndIf
\If{$c$} 
    \State $\textit{sleep\_for\_seconds(1)}$
\EndIf
\If{$d$} 
    \State $\textit{sleep\_for\_seconds(2)}$
\EndIf
\If{$a \textit{ and } b$} 
    \State $\textit{sleep\_for\_seconds(2)}$
\EndIf
\If{$c \textit{ and } d$} 
    \State $\textit{sleep\_for\_seconds(0)}$
\EndIf

\EndProcedure
\end{algorithmic}
\end{algorithm}


In Algorithm \ref{alg:performanceExample} we see a simple code snippet with some features that affect the runtime in different ways, in line 
2 four features a,b,c and d are defined, each of them can be either True or False depending on the configuration chosen. The code in
line 5,7,9,11 will is exectued only if the corresponding features are active. If this is the case, the system sleeps for the specified time. In lines
12 and 14, we have two feature interactions where $a$ and $b$ must be active, so in order for line 13 to executed, we would attribute the time spend
in line 13 to the feature interaction a,b and not to either feature alone.
The performance-influence model for our system would look like thiss:

\begin{equation*}
    \Pi = 1 + 1 \cdot c(a) + 2\cdot c(b) + 1\cdot c(c) + 2\cdot c(d) + 2 \cdot c(a)\cdot c(b) + 0\cdot c(c) \cdot c(d)
\end{equation*}

For simplicity, let us assume that the execution of the code takes no time at all and that no additional time is spent on any feature except 
the specified in the $textit{slee\_for\_seconds}$ function. The constant 1 here refers to $\beta_0$, which is the time we spend in our base 
feature in line 2. If we decide to turn on features a, b and c the equation would look like this:

\begin{align*}
    \Pi &= 1 + 1 \cdot c(a) + 2\cdot c(b) + 1\cdot c(c) + 2\cdot c(d) + 2 \cdot c(a)\cdot c(b) + 0\cdot c(c) \cdot c(d) \\
    \Pi &= 1 + 1 \cdot 1 + 2 \cdot 1 + 1 \cdot 1 + 2 \cdot 0 + 2 \cdot 1 \cdot 1 + 0 \cdot 1 \cdot 0 \\
    \Pi &= 1 + 1 + 2 + 1 + 2 + 2 \\
    \Pi &= 9
\end{align*}

Thus, for the configuration containing a,b,c our performance-influence model would yield an expected time of 9 second.

We build a performance-influence model from the date collected by both approaches, the challenge being to produce accurate data
to produce an accurate model. Since we apply both approaches to the same data, we expect them to produce a comparable performance-influence model.
We expain how we collect and work with the data in Chapter \ref{ch:Blackbox}%TO DO REF TO WHITEBOX AND BLACKBOX