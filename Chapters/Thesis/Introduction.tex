%************************************************
\chapter{Introduction}\label{ch:introduction}
%************************************************

Nowadays, all modern software systems are configurable and offer a large variety of functionality to satisfy multiple 
interest groups with a single software system. These functionalities are represented as features that the user chooses by selecting the 
configuration options corresponding to that feature. With the increasing complexity of modern configurable software systems, 
the number of features the system contains also increases.

An example of such a configurable software system would be the Linux kernel, whose
code base itself contains over 6‘000‘000 lines of code with more than 10‘000 features~\cite{Linux-Kernel}.
All these features implement functionality, which allows the user to select the features he
wants to create an operating system that meets the user’s needs.

All these features make it increasingly difficult to understand how much a feature influences the overall runtime of a configurable software 
analysis. Furthermore, all these features might interact with one another, which also changes the software system's behavior and runtime. 
To identify to which degree these features and their interactions influence the system, previous research introduced two different analyses: 
white-box and black-box.

For the black-box analysis, 
it is not necessary to have access to the source code since we only measure the time spent when executing the system using different features. 
We repeat this measurement process for all the features and feature interactions we want to measure. 
Afterwards, we need to infer the time spent for each feature. For this, we can use machine learning methods like multiple linear regression. 

If we have access to the system's source code, we can use a white-box analysis that uses this information to archive a more fine-grained 
view of the system. A white-box analysis uses different strategies to analyze the time spent inside features or feature interactions, 
such as a taint analysis.

Our primary interest point is to compare both analyses, however, they both produce a different form of data as output. 
Therefore, we need to find a model which makes the data comparable. 
Thus, we introduce {\perfInfluenceModel}s a way to model configurable software systems by assigning the expected influence to each feature 
and feature interaction. When giving the {\perfInfluenceModel} a selection of features, it maps them to the estimated runtime for the system.

\subsection{Goal of this Thesis}
In this work, we compare white-box analysis and black-box analysis with each other. 
We investigate how accurately both analyses can identify the influence of features and their interactions,
 to identify potential advantages and disadvantages of one method over the other. 
 Additionally, we are interested in how similar the {\perfInfluenceModel}s we build are. 
 Since they are built for the same system, we want to know if the models agree.

We collect all these interest points to form the following research questions that we answer during this thesis: 

\begin{itemize}\label{researchQuestions}
    \item[RQ1]: How accurately does white-box and black-box models detect feature and feature interactions? 
    \item[RQ2]: Do performance models created by our white-box and black-box attribute the same influence to each feature?
\end{itemize}

\subsection{Contributions}
We answer both research questions we design five different configurable software systems. 
In addition, we use both analyses on the real world compression tool \textsc{XZ}.