%************************************************
\chapter{Introduction}\label{ch:introduction}
%************************************************

Modern software systems are designed to be configurable, we want to provide flexibility to the useer by offering them to turn functionality on 
and off. 
We also expect a configurable software system to satisfy the demand of multiple users by offering a single software system that 
contains multiple features. \cite{Feature-Oriented-Software-Product-Lines}. 

An example of such a software system would be the Linux kernel, whose code base itself contains over 6.000.000 lines of code containing more 
than 1.000 optional features \footnote{https://www.kernel.org/doc/html/v4.14/admin-guide/kernel-parameters.html}. 
All these optional features allow you to create an operating system that meets your needs. To effectively analyze such systems,
we present two different model - a white-box and a black box-model.

This results in a large number of features that affect the system in different ways, to keep track of all these features and their interactions,
we will use a feature model, which is essentially a tree that can contain different kinds of constraints to visualize the relationships 
between features in a configurable system \cite{KangFeatureOrientedDomain1990}.

In the black-box model, we have only one system that receives an input, in our case a selection of several different features.
The system is then be executed with our configuration. During execution, we can meassure various metrics, such as memory and energy
consumption. However, we will focus on the meassured exectuion time.

In our white-box model, we have more information because we know the inner workings of the
system itself, i.e., we know which code contributes to which feature and can therefore meassures the time we spend in each feature by summing
up the time spent in the code when its executed.

Our two models generate different types of data that we still need to compare and evaluate, for which we use performance-
influence models. These models represent our configurable system as a polynomial, where each term represents either a feature or an 
interaction of features \cite{Performance-influence-models-for-highly-configurable-systems}. To build these models we use the data generated 
by the white-box and black-box models.

To show the validity of our models, we establish a ground truth. To do this, we design a small configurable system to test both of our 
models. This system will contain several features, some of which interact with each other in different ways. 
Since we developed this system ourselves, we know how each feature should impact the runtime of our system, so we create
a baseline performance-influence model to compare our models against.

After confirming the validity of our models, we will apply both models to real world systems, such as the compression tool
XZ. We will apply both models on the same system and data then reapeat the experiment 30 times for each configuration to reduce external
factors such as meassuremest noise.

\section{Goals of this thesis}
In this thesis the main focus is about comparing performance-influence models between white-box and black-box models. 
Before we can even compare these models, we need to check whether they are able to detect interactions between features, 
and if so, how accurate they are. If they can identify the interactions between the features, we can start to compare them. 

First and foremost, we are interested in whether both models come to the same conclusions, after all, they have both analyzed the same system. 
If they reach the same conclusion, we can already see that it is feasible to use either of the two models to analyze a system, 
but from there, we still work out the advantages and trade-offs between the models so that the user can choose the one that meets 
his needs. If they do not reach the same conclusion, we analyze the reason for the differences between them and examine whether one model 
performs particularly poorly in certain cases and why this is so. We answer the following research questions:\\\\

\noindent \textbf{RQ1}: How accurately does white-box and black-box models detect feature interactions? \\
\noindent \textbf{RQ2}: Do performance models created by our white-box and black-box reach the same conclusion?\\
\noindent \textbf{RQ3}: Can we identify the reasons for similarities or differences between performance models?\\
