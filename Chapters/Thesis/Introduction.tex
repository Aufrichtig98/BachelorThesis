%************************************************
\chapter{Introduction}\label{ch:introduction}
%************************************************

%Introduction in configureable systems why they exist
Modern software systems are designed to be configurable; they offer users flexibility by providing them to turn functionality on and off.
This flexibility allows a configurable software system to satisfy the demand of multiple users by offering a single software system that 
contains multiple features~\cite{Feature-Oriented-Software-Product-Lines}. 

%Example configureable system linux
An example of such a configurable software system would be the Linux kernel, whose code base itself contains over 6`000`000 lines of code consist of more 
than 10`000 features~\cite{Linux-Kernel}. 
All these optional features allow the user to create an operating system that meets his needs. 

%Feature model to show overwiew over all these features
All these features affect the system in different ways. To keep track of all features and their interactions, we use a feature model, 
that describes how features interact with another.
Inside the feature model, there can be different kinds of constraints to visualize the relationships 
between features in a configurable system~\cite{Feature-Oriented-Software-Product-Lines}. 

%After modeling system we want to analyze we explain that we are intrested in the performance of each feature and use 2 analyses
Now that we have an overview of the system we are interested in what capacity each feature or feature interaction influences the system,
To identify these influences we present two different analyses: a white-box analysis and a black-box analysis.
However, before this, we analyze the system's configuration space and select the crucial features of the system.
Then, we create configurations from the selected features that contain the interactions we want to observe.

%Explains black-box roughly
In the black-box analysis, we run the system with each configuration as input, and during execution, we can measure various metrics, such as memory and energy consumption.
Whereas we will focus on measuring the execution time. We will use the measurement we collect to learn each feature's influence on the system 
using multiple linear regression.

%Explains white-box roughly
In our white-box analysis, we have more information because we have access to the system's source code.
We use an analysis that helps us determine which parts of the code are influenced by which feature.
Then during execution, we can measure the time spent inside these features.

%Reasoning why we use perf inf models
Now both our analyses generate different types of data that are different. 
Therefore, we need to transform this data into a model that we can then use to evaluate both analyses by comparing these models; 
for this, we use \perfInfluenceModel. These models represent our configurable system as a polynomial, 
where each term represents either a feature or an interaction of features~\cite{Performance-influence-models-for-highly-configurable-systems}.
We build these models by using the data generated by the white-box analysis and black-box analysis. 

%Reason to why we use a ground truth
To show the validity of our models, we establish a ground truth. To do this, we design a small configurable system to test both of our 
models. This system will contain several features, some of which interact with each other in different ways. 
Since we developed this system ourselves, we know how each feature should impact the runtime of our system, so we create
a baseline performance-influence model to compare our models against.


\section{Goals of this thesis}
%Simplea goal, what we want to archieve in the thesis
In this thesis, we aim to determine if white-box and black-box analyzes can identify the influence of the different features and 
feature interactions on a configurable system.
Furthermore, we are interested in the differences of the \perfInfluenceModel{s} built with data generated by each analysis.

%Describing research question
We are interested in whether both models can correctly identify the influence of each feature.
If they reach the same conclusion, i.e., they attribute the same influence to each feature. 
We know it is feasible to use either of the two models to analyze a system, but from there, 
we still work out the advantages and trade-offs between the models so that the user can choose the one that meets his needs.
Suppose they reach a different conclusion. In that case, we analyze the reason for the 
differences between them and examine whether one model performs particularly poorly in some instances and why this is so.

We collect all these interest points to form the following research questions:

\begin{itemize}\label{researchQuestions}
\item[RQ1]: How accurately does white-box and black-box models detect feature interactions? 
\item[RQ2]: Do performance models created by our white-box and black-box attribute the same influence to each feature?
\item[RQ3]: What are the reasons for similarities or differences between performance models?
\end{itemize}
