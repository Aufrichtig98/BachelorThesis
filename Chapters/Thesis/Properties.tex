\section{Functional and Non-functional Properties}\label{ch:properties}

When analyzing a configurable system, we distinguish between what a system can do and how the system archives that goal; the former,
refers to the functional properties, while the latter describes the non-functional properties of the system. \cite{non-functional-Properties}

So when we talk about functional properties, we mean everything a system can do, including which problem it solves and what
features the system provides for us, the user. 
For example in \autoref{fig:xz} we can see two main features of XZ, compression and decompression are two main components of the system,
they refer to the functionality XZ provides.

While functional properties refer to what a system can do, non-functional properties refer to how that functionality is achieved. 
To do so we can measure most of the nun-functional properties, such as performance, memory consumption, CPU usage and energy consumption. 
However, not all non-functional properties are quantifiable. Some relate to the quality of a system that we can not easily measure, such as
how secure a system is designed or how high the code quality is in regard to code readability or documentation. \cite{non-functional-Properties}