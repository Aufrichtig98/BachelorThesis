\section{Functional and Non-functional Properties}\label{ch:properties}

%Functional properties with example
When analyzing a configurable system, we distinguish between what a system can do and how the system archives that goal;
the former refers to the functional properties, while the latter describes the non-functional properties of the system\cite{non-functional-Properties}.
When we talk about functional properties, we mean everything a system can do, including which problem it solves and what features the system provides us, 
the user.
For example, in \autoref{fig:xz}, we can see the features of \textsc{XZ}, \textit{Compression}, 
and \textit{Decompression}; they refer to the functionality \textsc{XZ} provides.

%Non-function properties with example
While functional properties refer to what a system can do, non-functional properties refer to how or in which circumstance
that functionality is achieved\cite{non-functional-Properties}. 
To do so we can measure most of the nun-functional properties, such as performance, memory consumption, CPU usage and energy consumption. 
However, not all non-functional properties are quantifiable. Some relate to the quality of a system that we can not easily measure, such as
how secure a system is or how high the code quality is in regard to code readability or documentation\cite{non-functional-Properties}.