\subsection{Features and Configurations}\label{feature-config}

%What is a feature
During the years there have been many definitions of what a \emph{feature} is, on one side, features are used as a means of communication between
the different stakeholder of a system, where on the other hand, a feature is defined as an implementation-level concept. 
To unify both usages Apel et. al. introduced the following definition~\cite{Feature-Oriented-Software-Product-Lines}:

%Definition
\begin{definition}
    "A structure that extends and modifies the structure of a
    given program in order to satisfy a stakeholder's requirement, to implement and
    encapsulate a design decision, and to offer a configuration option."     
\end{definition}

%Example of feature on xz an definition of configuartion
Thus, a feature is both an abstract concept that refers to particular functionality of a system and the implementation of that functionality.
In our example \autoref{fig:xz} both, \textit{Compression} and \textit{Decompression}, are unique features 
that refer to a piece of functionality of \textsc{XZ} and the implementation of the functionality. 

%Featuere Interaction
Inside a configurable software system, features are not independent of one another. 
Most of the time, features influence the behavior of different features. 
When this happens, we say that these features interact with each other. 
Due to this, we are interested in the degree to which a feature interaction influences the system.

%Numeric and binary feature
We differentiate between \emph{binary} features and \emph{numeric} features. A binary feature can be either selected or deselected.
Commonly, when we select a binary feature we represent it with $1$ and $0$ otherwise. 
A numeric feature is a feature which, if selected, requires a numerical value that specifies a different behavior of that feature.
These can have various meanings depending on the feature it implements. 
For \autoref{fig:xz}, \textit{Decompression} could be modeled as a binary feature, since we have the option to decrypt a file or not. On the contrary, 
\textit{Compression} could be implemented as a numeric feature, where the numeric value represents the quality of compression we want. 

%What is a Configuration option
A configuration option is a predefined way for developers to change the functionality of the configurable system.
These options allow us to select features we want to include or exclude. Therefore, a configuration is a set of configuration options.  
We call a configuration valid as long as the selection of configuration option is allowed by the system.

%What is a Configuration space
The configuration space refers to the set of all possible configurations. Some of these configurations can be invalid. 
As we cannot execute systems with invalid configurations in practice, our work focuses on valid configuration,
hence, when we refer to a configuration space we always mean the space of valid configurations, except otherwise noted.

\mycomment{ Appearently does not connect to the section
%Example
As an example, \textsc{XZ} offers the user different features such as \textit{Compression} and \textit{Decompression}, whereas 
the configuration options \textit{0-9} specify the degree of compression. An invalid configuration would be to select multiple 
degrees of compression.
}