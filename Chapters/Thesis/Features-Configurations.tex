\section{Features and Configurations}

In the previous section we already said that we want to be able to turn functionality on and off, to achieve this we use features.

During the years there have been many definitions to what a feature is, on one side, features are used as a means of communication between
the different stakeholder of a system, where on the other hand, a feature is defined as an implementation-level concept. 

To incorporate both concepts Apel et.al. \cite[p. 18]{Feature-Oriented-Software-Product-Lines} introduced us to the following definition:

"a structure that extends and modifies the structure of a
given program in order to satisfy a stakeholder's requirement, to implement and
encapsulate a design decision, and to offer a configuration option"

Thus, a feature is both an abstract concept that refers to particular functionality of a system and the implementation of that functionality.
In our example \ref{fig:xz} both, encryption and decryption, are unique features, they refer to a piece of functionality of the system and the
implementation. 
A configuration is a set of features, where the features selected in the configuration decide which functionality of the system is turned on or off.
The set of all valid configurations of a system is called the configuration space, it depicts the whole functionality of the system.
In \ref{fig:xz} the configuration we select, decides if we want to have encryption or decryption enabled.