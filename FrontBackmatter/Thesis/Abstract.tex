%*******************************************************
% Abstract
%*******************************************************
%\renewcommand{\abstractname}{Abstract}
\pdfbookmark[1]{Abstract}{Abstract}
% \addcontentsline{toc}{chapter}{\tocEntry{Abstract}}
\begingroup
\let\clearpage\relax
\let\cleardoublepage\relax
\let\cleardoublepage\relax

\chapter*{Abstract}
Nearly all modern software systems are configurable. An important reason developers introduce configurability is end-user flexibility using configurable options; 
the user can tailor the system to their needs. 

However, nowadays, configurable systems offer various configurable options, resulting in many unique configurations. 
All of these different configurations result in other behaviors of the system; stakeholders are interested in which degree 
these behaviors influence various system metrics. One way to obtain these metrics is by examining the system's run time under each configuration. 
 
In the past different black-box and white-box analyses have been proposed to analyze the performance of configurable systems. In this 
thesis, we use a white-box analysis and a black-box analysis to identify the influence of each feature on a configurable system.
We use the data generated by those approaches to build performance-influence models and use them to analyze and compare both analyses. 

We use both analyses on the compression tool \textsc{XZ} and five different systems built by us.  %Generelle erkentnisse 

\vfill

\endgroup

\vfill
