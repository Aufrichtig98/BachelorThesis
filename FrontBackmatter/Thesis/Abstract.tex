%*******************************************************
% Abstract
%*******************************************************
%\renewcommand{\abstractname}{Abstract}
\pdfbookmark[1]{Abstract}{Abstract}
% \addcontentsline{toc}{chapter}{\tocEntry{Abstract}}
\begingroup
\let\clearpage\relax
\let\cleardoublepage\relax
\let\cleardoublepage\relax

\chapter*{Abstract}
Nearly, all modern software systems are configurable, a major reasons for that is that we as the developer 
want to give the user the flexibility of configuring the system to their needs, by offering them to turn functionality on and off. 

However, nowadays configurable systems offer a various features which result in a large ammount of unique configurations. All of these different
configurations result in different running times for the system and stakeholder are intrested in what capacity each feature influcences
the run time of the system. To analyze the influence of these features we will introduce two different approaches, a white-box and black-box
approach. With the data generated by our Two approaches we will build performance-influence models and use them to analyze and compare both
approaches. 




\vfill

\endgroup

\vfill
