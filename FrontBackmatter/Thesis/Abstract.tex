%*******************************************************
% Abstract
%*******************************************************
%\renewcommand{\abstractname}{Abstract}
\pdfbookmark[1]{Abstract}{Abstract}
% \addcontentsline{toc}{chapter}{\tocEntry{Abstract}}
\begingroup
\let\clearpage\relax
\let\cleardoublepage\relax
\let\cleardoublepage\relax

\chapter*{Abstract}
Nearly all modern software systems are configurable. A significant reason developers introduce configurability is end-user flexibility with configurable options; 
the user can tailor the system to their needs, turning functionality on and off. 

However, nowadays, configurable systems offer various configurable options, resulting in many unique configurations. 
All of these different configurations result in other behaviors of the system that we want to analyze; stakeholders are interested in which degree 
these behaviors influence various system metrics. One way to obtain these metrics is by examining the system's running time under each configuration. 
 
In this thesis, we introduce two different analyses to identify the influence of each feature on a configurable system, a white-box analysis 
and a black-box analysis.
We use the data generated by those approaches to build performance-influence models and use them to analyze and compare both methods. 

\vfill

\endgroup

\vfill
