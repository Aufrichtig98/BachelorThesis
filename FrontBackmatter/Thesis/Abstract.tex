%*******************************************************
% Abstract
%*******************************************************
%\renewcommand{\abstractname}{Abstract}
\pdfbookmark[1]{Abstract}{Abstract}
% \addcontentsline{toc}{chapter}{\tocEntry{Abstract}}
\begingroup
\let\clearpage\relax
\let\cleardoublepage\relax
\let\cleardoublepage\relax

\chapter*{Abstract}
Nearly, all modern software systems are configurable, a major reasons for that is that we as the developer 
want to give the user the flexibility of configuring the system to their needs, by offering them to turn functionality on and off. 

However, nowadays configurable systems offer various features which result in a large ammount of unique configurations. All of these different
configurations result in different behaviours of the system, one of the metrics shareholder are intrested 
is the running times of system under each configuration and in what capacity each feature influcences
the running time. To analyze the influence of these features we will introduce two different approaches, a white-box and black-box
approach. We will use the data generated by those approaches to build performance-influence models and use them to analyze and compare both
approaches. 

\vfill

\endgroup

\vfill
