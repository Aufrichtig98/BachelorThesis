%*******************************************************
% Abstract
%*******************************************************
%\renewcommand{\abstractname}{Abstract}
\pdfbookmark[1]{Abstract}{Abstract}
% \addcontentsline{toc}{chapter}{\tocEntry{Abstract}}
\begingroup
\let\clearpage\relax
\let\cleardoublepage\relax
\let\cleardoublepage\relax

\chapter*{Abstract}
Nearly all modern software systems are configurable. To give the end-user flexibility, each system includes various configuration options.
However, it is not clear to the end-user how these configuration options might interact and influence properties such as performance. 
We are interested in whether a change in behavior is the result of single independent configuration options or due to these options
interacting with each other.
In previous research two different methods have emerged for identifying these influences: black-box analysis and white-box analysis.
To present the influence of each configuration option and interaction between them, we use {\perfInfluenceModel}, which we use as our foundation
to compare the results of both analyses.

In this thesis, we analyze 5 different configurable software systems designed by us and the compression tool \textsc{XZ} using both white-box analysis
and black-box analysis. We present a structured approach to compare white-box and black-box analyses. 
We found out that systems containing multicollinearity led to inaccurate results for both analyses. 
In addition, the white-box analysis could not identify most configuration options for \textsc{XZ}.
\vfill

\endgroup

\vfill
